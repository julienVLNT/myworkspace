
\title{Volumes finis et nouvelles conditions de saut pour les écoulements à l'interface fluide / poreux}
\author{Julien VALENTIN}
\date{\today}

\thispagestyle{empty}
	
	\begin{figure}[ht]
	   \minipage{0.70\textwidth}

			\includegraphics[scale=.06]{Images/Logo_SU.png}
			\label{Logo_SU}
	   \endminipage
	   \minipage{0.39\textwidth}
			\includegraphics[scale=.49]{Images/Logo_I2M.png}
			\label{Logo_I2M}
		\endminipage
	\end{figure}
	
	\begin{center}
	\vspace{1.5cm}
	\LARGE
	Mémoire de fin d'études, master mathématiques pour la modélisation de Sorbonne-Université 
	
	\vspace{0.8cm}
	\LARGE
	\'A l'Institut de Mathématiques de Marseille

	
	\vspace{3cm}	
	\Large
	\textbf{Etudes numériques d'écoulements de fluides visqueux purs et en milieux poreux}

	\vspace{4cm}
	\normalsize	
	%ETUDIANT \\
	\vspace{.3cm}
	\large
	\textbf{Julien VALENTIN}
	
	\vspace{1.7cm}
	\normalsize	
    ENCADR\'E PAR  \\
	\vspace{.3cm}
	\large
	\textbf{Philippe ANGOT}
	
	\vspace{1.3cm}

	\vfill
	\today
	\end{center}