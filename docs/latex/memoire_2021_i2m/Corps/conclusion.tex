\addcontentsline{toc}{chapter}{Conclusion}

\chapter*{Conclusion}

\noindent En cinq mois de stage, j'ai exploré différents aspects du travail de l'ingénieur en analyse numérique. Pour commencer, le choix des outils et des technologies permettant d'effectuer le calcul souhaité, ici la résolution de systèmes linéaires associés aux systèmes de type Stokes. Je me suis ouvert à une méthode de discrétisation des lois de conservations, les volumes finis, et aux concepts qui y sont associés : les objets que sont les maillages, et leur caractère admissible ou non ; les choix à effectuer lors de la discrétisation et éventuellement leur impact sur la simulation, je pense au choix de la représentation de la grandeur étudiée, valeur moyenne, valeur ponctuelle, ou aure, les choix de discrétisation des flux : une fois obtenue l'intégrale surfacique, il faut à nouveau approcher cette intégrale, plutôt que de discrétiser par le schéma centré d'ordre 1, nous aurions pu regarder une discrétisation de ces intégrales par éléments finis, etc. Une fois le maillage choisi et le problème discrétisé, il s'agit de montrer l'existence et l'unicité d'une solution à ce nouveau problème. Enfin, il s'agit d'étudier la stabilité des flux, par l'estimation point par point d'abord de l'erreur entre une solution classique et une solution numérique, calculée par le schéma, puis son estimation en norme. Il ne faut pas oublier de se pencher sur la consistance de l'opérateur discret, comprendre si son inverse existe et reste borné pour de petits pas de maillages. Alors on peut établir la convergence et, si une solution analytique est connue, implémenter le schéma et valider le code produit en analysant l'erreur en fonction du pas de maillage pour différentes normes discrètes. Bien que certains faits et résultats sont très classiques, les méthodes de volumes finis sont un domaine d'étude en soi que j'espère pouvoir creuser dans la suite de ma vie professionnelle. 

Ensuite est venu le temps d'apprendre la mécanique des fluides, et de consolider mes connaissances en analyse. J'ai probablement passé trop de temps sur ces points mais qui me paraissent pourtant fondamentaux du point de vue la méthode. Il me semble que l'établissement du caractère bien posé d'un problème est incontournable avant de passer à la simulation. Cette partie sur le problème de Stokes m'a permis de fixer les étapes de l'analyse d'un problème et de mobiliser les arguments d'analyse fonctionnelle étudiés dans les différents cours que j'ai pu suivre cette année afin de comprendre le développement de cette analyse. Du reste, j'ai appris au cours de ce stage la condition $\inf \sup$ et les conditions de compatibilités pour le problème de Stokes. C'est ce point qui fait que ce n'est pas, pour moi, simplement de la redite mais un véritable pas en avant dans mes connaissances.

La suite du mémoire a été l'occasion de dérouler les méthodes et d'asseoir les découvertes vues dans les deux premières parties. Le système de Brinkman en particulier apparaît comme étant un système de Stokes généralisé, avec un terme d'inertie, appelé terme de Darcy dans le cas des milieux poreux. On peut alors montrer l'existence et l'unicité d'une solution par une méthode lagrangienne, utiliser les résultats d'optimisation dans les espaces de Banach pour en rédiger les preuves, et se rapporter sans retenue aux ouvrages contenant les résultats nécessaires.

Enfin, l'étude de la superposition des deux couches, fluides et poreuses, n'aura été qu'effleurée, sur les deux dernières semaines. Les mêmes qui ont vu naître ce mémoire. Je n'y ai pas passé assez de temps pour dire quoi que ce soit d'intelligent à ce sujet. Ici de nouvelles difficultés apparaissent avec la notion de saut, ce qui implique l'apparition de nouveaux espaces fonctionnels donc un nouveau cadre pour discuter l'existence et l'unicité, de nouvelles difficultés dans le choix du maillage, puis une nouvelle discrétisation du problème.